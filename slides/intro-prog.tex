\documentclass{beamer}

\usepackage[utf8x]{inputenc}
\usepackage{listings}


\title{Notions de programmation}
\author{Zeste de Savoir}
\usetheme{zestedesavoir}
\begin{document}

\begin{frame}
  \titlepage
\end{frame}

\section{Notions}

\begin{frame}
    \frametitle{Découverte de l'interpréteur}
    Faire des calculs mathématiques simple
    \begin{itemize}
    \item \(1+1\)
    \item \(6\times 7\)
    \item \(4-2\)
    \item \((6-4)\times 2\)
    \item \(3.6 / 2\)
    \end{itemize}

    Exercice : calculer le triple de 5,2
\end{frame}

\begin{frame}
    \frametitle{Découverte de l'interpréteur}
    \textbf{Variable} : boite dans laquelle on peut mettre une valeur
    \begin{itemize}
        \item Taper \texttt{x=3} puis taper \texttt{x}
        \item Taper \texttt{x+1} puis taper \texttt{x}
        \\ La valeur de \texttt{x} a-t-elle changée ?
        \item Taper \texttt{y=4} puis taper \texttt{y=y+2}
        \\ La valeur de \texttt{y} a-t-elle changée ?
    \end{itemize}
\end{frame}

\begin{frame}[fragile]
    \frametitle{Appeler une fonction}
    \textbf{Fonction} : morceau de code que quelqu'un d'autre a déjà écrit pour vous, qu'on appelle avec son nom pour ne pas le ré-écrire.
    \medbreak
    \begin{lstlisting}[language=python]
    print("Bonjour !")
    \end{lstlisting}
\end{frame}

\begin{frame}
  \frametitle{Votre premier programme}
  \begin{block}{Exercice}
    Écrivez un programme qui calcule le double d'un nombre. Pour cela, votre programme va prendre en entrée une variable \texttt{x} et affichera le double de \texttt{x}.
  \end{block}
  Par exemple si \texttt{x=3}, le programme doit afficher \texttt{6}.
\end{frame}

\begin{frame}[fragile]
  \frametitle{Correction}
  \begin{lstlisting}[language=python]
    x = 3
    x= 2*x
    print(x)
  \end{lstlisting}
\end{frame}

\begin{frame}[fragile]
    \frametitle{Conditions}
    Si ton age est supérieur ou égal à 18 ans, alors tu es majeur !
    \begin{lstlisting}[language=python]
    age = 23\\
    if age $>=$ 18:\\
    ... print("Tu es majeur")
    \end{lstlisting}

    \medbreak
    Liste des tests possibles :

    \texttt{>}  : supérieur à\\
    \texttt{>=} : supérieur ou égal à\\
    \texttt{<}  : inférieur à\\
    \texttt{<=} : inférieur ou égal à\\
    \texttt{==} : égal à\\
    \texttt{!=} : différent de\\

\end{frame}

\begin{frame}[fragile]
    \frametitle{Conditions (ET et OU)}
    Si tu as plus que 11 ans mais moins que 18, alors tu es adolescent.
    \medbreak
    \begin{lstlisting}[language=python]
    age = 14
    if age $>$ 11 $\&\&$ age $<$ 18:
    ... print("Tu es adolescent ! ")
    \end{lstlisting}

    \medbreak
    \begin{itemize}
    \item     \texttt{\&\&} correspond au ``ET'' logique \\
    \item     \texttt{||} correspond au ``OU'' logique
    \end{itemize}
\end{frame}

\begin{frame}[fragile]
    \frametitle{Conditions (avec else)}
    Si tu as au moins 18 ans alors tu es majeur. Sinon, tu es mineur :
    \medbreak
    \begin{lstlisting}[language=python]
    age = 23
    if age $>=$ 18:
    ... print("Tu es majeur")
    else:
    ... print("Tu es mineur")
    \end{lstlisting}

\end{frame}

\begin{frame}[fragile]
    \frametitle{Boucles}

    Pour tout les ages de 5 à 25 ans, dire si la personne est majeur.
    \begin{lstlisting}[language=python]
    for age in range(5,26):
    ... print(age)
    ... if age $>$ 18:
    ... ... print(" = majeur \n")
    ... else:
    ... ... print(" = mineur \n")
    \end{lstlisting}
\end{frame}

\end{document}